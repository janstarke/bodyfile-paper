\documentclass[a4paper]{article}

\newcommand{\bodyfile}{\texttt{bodyfile}}

\title{Improving the \bodyfile\ format: a suggestion}
\author{Jan Starke <jan.starke@t-systems.com>}
\date{\today}



\begin{document}

\maketitle

\begin{abstract}
    In this paper we describe the famous \bodyfile\ format and, especially, a lot of its shortcomings. But because this format is widely known and used, we discuss the benefits of \bodyfile\ and take a look at the requirements that an improved fileformat should fulfill.
\end{abstract}

\section{The \bodyfile\ format}

A lot of forensic tools collect data which are related with time information. We can see sich information as sets of events. Those data must be related to each other, which normally means that they are ordered chronologically, so that one can which event occured after another.
\end{document}